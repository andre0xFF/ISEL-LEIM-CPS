
% Default to the notebook output style

    


% Inherit from the specified cell style.




    
\documentclass[11pt]{article}

    
    
    \usepackage[T1]{fontenc}
    % Nicer default font (+ math font) than Computer Modern for most use cases
    \usepackage{mathpazo}

    % Basic figure setup, for now with no caption control since it's done
    % automatically by Pandoc (which extracts ![](path) syntax from Markdown).
    \usepackage{graphicx}
    % We will generate all images so they have a width \maxwidth. This means
    % that they will get their normal width if they fit onto the page, but
    % are scaled down if they would overflow the margins.
    \makeatletter
    \def\maxwidth{\ifdim\Gin@nat@width>\linewidth\linewidth
    \else\Gin@nat@width\fi}
    \makeatother
    \let\Oldincludegraphics\includegraphics
    % Set max figure width to be 80% of text width, for now hardcoded.
    \renewcommand{\includegraphics}[1]{\Oldincludegraphics[width=.8\maxwidth]{#1}}
    % Ensure that by default, figures have no caption (until we provide a
    % proper Figure object with a Caption API and a way to capture that
    % in the conversion process - todo).
    \usepackage{caption}
    \DeclareCaptionLabelFormat{nolabel}{}
    \captionsetup{labelformat=nolabel}

    \usepackage{adjustbox} % Used to constrain images to a maximum size 
    \usepackage{xcolor} % Allow colors to be defined
    \usepackage{enumerate} % Needed for markdown enumerations to work
    \usepackage{geometry} % Used to adjust the document margins
    \usepackage{amsmath} % Equations
    \usepackage{amssymb} % Equations
    \usepackage{textcomp} % defines textquotesingle
    % Hack from http://tex.stackexchange.com/a/47451/13684:
    \AtBeginDocument{%
        \def\PYZsq{\textquotesingle}% Upright quotes in Pygmentized code
    }
    \usepackage{upquote} % Upright quotes for verbatim code
    \usepackage{eurosym} % defines \euro
    \usepackage[mathletters]{ucs} % Extended unicode (utf-8) support
    \usepackage[utf8x]{inputenc} % Allow utf-8 characters in the tex document
    \usepackage{fancyvrb} % verbatim replacement that allows latex
    \usepackage{grffile} % extends the file name processing of package graphics 
                         % to support a larger range 
    % The hyperref package gives us a pdf with properly built
    % internal navigation ('pdf bookmarks' for the table of contents,
    % internal cross-reference links, web links for URLs, etc.)
    \usepackage{hyperref}
    \usepackage{longtable} % longtable support required by pandoc >1.10
    \usepackage{booktabs}  % table support for pandoc > 1.12.2
    \usepackage[inline]{enumitem} % IRkernel/repr support (it uses the enumerate* environment)
    \usepackage[normalem]{ulem} % ulem is needed to support strikethroughs (\sout)
                                % normalem makes italics be italics, not underlines
    

    
    
    % Colors for the hyperref package
    \definecolor{urlcolor}{rgb}{0,.145,.698}
    \definecolor{linkcolor}{rgb}{.71,0.21,0.01}
    \definecolor{citecolor}{rgb}{.12,.54,.11}

    % ANSI colors
    \definecolor{ansi-black}{HTML}{3E424D}
    \definecolor{ansi-black-intense}{HTML}{282C36}
    \definecolor{ansi-red}{HTML}{E75C58}
    \definecolor{ansi-red-intense}{HTML}{B22B31}
    \definecolor{ansi-green}{HTML}{00A250}
    \definecolor{ansi-green-intense}{HTML}{007427}
    \definecolor{ansi-yellow}{HTML}{DDB62B}
    \definecolor{ansi-yellow-intense}{HTML}{B27D12}
    \definecolor{ansi-blue}{HTML}{208FFB}
    \definecolor{ansi-blue-intense}{HTML}{0065CA}
    \definecolor{ansi-magenta}{HTML}{D160C4}
    \definecolor{ansi-magenta-intense}{HTML}{A03196}
    \definecolor{ansi-cyan}{HTML}{60C6C8}
    \definecolor{ansi-cyan-intense}{HTML}{258F8F}
    \definecolor{ansi-white}{HTML}{C5C1B4}
    \definecolor{ansi-white-intense}{HTML}{A1A6B2}

    % commands and environments needed by pandoc snippets
    % extracted from the output of `pandoc -s`
    \providecommand{\tightlist}{%
      \setlength{\itemsep}{0pt}\setlength{\parskip}{0pt}}
    \DefineVerbatimEnvironment{Highlighting}{Verbatim}{commandchars=\\\{\}}
    % Add ',fontsize=\small' for more characters per line
    \newenvironment{Shaded}{}{}
    \newcommand{\KeywordTok}[1]{\textcolor[rgb]{0.00,0.44,0.13}{\textbf{{#1}}}}
    \newcommand{\DataTypeTok}[1]{\textcolor[rgb]{0.56,0.13,0.00}{{#1}}}
    \newcommand{\DecValTok}[1]{\textcolor[rgb]{0.25,0.63,0.44}{{#1}}}
    \newcommand{\BaseNTok}[1]{\textcolor[rgb]{0.25,0.63,0.44}{{#1}}}
    \newcommand{\FloatTok}[1]{\textcolor[rgb]{0.25,0.63,0.44}{{#1}}}
    \newcommand{\CharTok}[1]{\textcolor[rgb]{0.25,0.44,0.63}{{#1}}}
    \newcommand{\StringTok}[1]{\textcolor[rgb]{0.25,0.44,0.63}{{#1}}}
    \newcommand{\CommentTok}[1]{\textcolor[rgb]{0.38,0.63,0.69}{\textit{{#1}}}}
    \newcommand{\OtherTok}[1]{\textcolor[rgb]{0.00,0.44,0.13}{{#1}}}
    \newcommand{\AlertTok}[1]{\textcolor[rgb]{1.00,0.00,0.00}{\textbf{{#1}}}}
    \newcommand{\FunctionTok}[1]{\textcolor[rgb]{0.02,0.16,0.49}{{#1}}}
    \newcommand{\RegionMarkerTok}[1]{{#1}}
    \newcommand{\ErrorTok}[1]{\textcolor[rgb]{1.00,0.00,0.00}{\textbf{{#1}}}}
    \newcommand{\NormalTok}[1]{{#1}}
    
    % Additional commands for more recent versions of Pandoc
    \newcommand{\ConstantTok}[1]{\textcolor[rgb]{0.53,0.00,0.00}{{#1}}}
    \newcommand{\SpecialCharTok}[1]{\textcolor[rgb]{0.25,0.44,0.63}{{#1}}}
    \newcommand{\VerbatimStringTok}[1]{\textcolor[rgb]{0.25,0.44,0.63}{{#1}}}
    \newcommand{\SpecialStringTok}[1]{\textcolor[rgb]{0.73,0.40,0.53}{{#1}}}
    \newcommand{\ImportTok}[1]{{#1}}
    \newcommand{\DocumentationTok}[1]{\textcolor[rgb]{0.73,0.13,0.13}{\textit{{#1}}}}
    \newcommand{\AnnotationTok}[1]{\textcolor[rgb]{0.38,0.63,0.69}{\textbf{\textit{{#1}}}}}
    \newcommand{\CommentVarTok}[1]{\textcolor[rgb]{0.38,0.63,0.69}{\textbf{\textit{{#1}}}}}
    \newcommand{\VariableTok}[1]{\textcolor[rgb]{0.10,0.09,0.49}{{#1}}}
    \newcommand{\ControlFlowTok}[1]{\textcolor[rgb]{0.00,0.44,0.13}{\textbf{{#1}}}}
    \newcommand{\OperatorTok}[1]{\textcolor[rgb]{0.40,0.40,0.40}{{#1}}}
    \newcommand{\BuiltInTok}[1]{{#1}}
    \newcommand{\ExtensionTok}[1]{{#1}}
    \newcommand{\PreprocessorTok}[1]{\textcolor[rgb]{0.74,0.48,0.00}{{#1}}}
    \newcommand{\AttributeTok}[1]{\textcolor[rgb]{0.49,0.56,0.16}{{#1}}}
    \newcommand{\InformationTok}[1]{\textcolor[rgb]{0.38,0.63,0.69}{\textbf{\textit{{#1}}}}}
    \newcommand{\WarningTok}[1]{\textcolor[rgb]{0.38,0.63,0.69}{\textbf{\textit{{#1}}}}}
    
    
    % Define a nice break command that doesn't care if a line doesn't already
    % exist.
    \def\br{\hspace*{\fill} \\* }
    % Math Jax compatability definitions
    \def\gt{>}
    \def\lt{<}
    % Document parameters
    \title{report}
    
    
    

    % Pygments definitions
    
\makeatletter
\def\PY@reset{\let\PY@it=\relax \let\PY@bf=\relax%
    \let\PY@ul=\relax \let\PY@tc=\relax%
    \let\PY@bc=\relax \let\PY@ff=\relax}
\def\PY@tok#1{\csname PY@tok@#1\endcsname}
\def\PY@toks#1+{\ifx\relax#1\empty\else%
    \PY@tok{#1}\expandafter\PY@toks\fi}
\def\PY@do#1{\PY@bc{\PY@tc{\PY@ul{%
    \PY@it{\PY@bf{\PY@ff{#1}}}}}}}
\def\PY#1#2{\PY@reset\PY@toks#1+\relax+\PY@do{#2}}

\expandafter\def\csname PY@tok@w\endcsname{\def\PY@tc##1{\textcolor[rgb]{0.73,0.73,0.73}{##1}}}
\expandafter\def\csname PY@tok@c\endcsname{\let\PY@it=\textit\def\PY@tc##1{\textcolor[rgb]{0.25,0.50,0.50}{##1}}}
\expandafter\def\csname PY@tok@cp\endcsname{\def\PY@tc##1{\textcolor[rgb]{0.74,0.48,0.00}{##1}}}
\expandafter\def\csname PY@tok@k\endcsname{\let\PY@bf=\textbf\def\PY@tc##1{\textcolor[rgb]{0.00,0.50,0.00}{##1}}}
\expandafter\def\csname PY@tok@kp\endcsname{\def\PY@tc##1{\textcolor[rgb]{0.00,0.50,0.00}{##1}}}
\expandafter\def\csname PY@tok@kt\endcsname{\def\PY@tc##1{\textcolor[rgb]{0.69,0.00,0.25}{##1}}}
\expandafter\def\csname PY@tok@o\endcsname{\def\PY@tc##1{\textcolor[rgb]{0.40,0.40,0.40}{##1}}}
\expandafter\def\csname PY@tok@ow\endcsname{\let\PY@bf=\textbf\def\PY@tc##1{\textcolor[rgb]{0.67,0.13,1.00}{##1}}}
\expandafter\def\csname PY@tok@nb\endcsname{\def\PY@tc##1{\textcolor[rgb]{0.00,0.50,0.00}{##1}}}
\expandafter\def\csname PY@tok@nf\endcsname{\def\PY@tc##1{\textcolor[rgb]{0.00,0.00,1.00}{##1}}}
\expandafter\def\csname PY@tok@nc\endcsname{\let\PY@bf=\textbf\def\PY@tc##1{\textcolor[rgb]{0.00,0.00,1.00}{##1}}}
\expandafter\def\csname PY@tok@nn\endcsname{\let\PY@bf=\textbf\def\PY@tc##1{\textcolor[rgb]{0.00,0.00,1.00}{##1}}}
\expandafter\def\csname PY@tok@ne\endcsname{\let\PY@bf=\textbf\def\PY@tc##1{\textcolor[rgb]{0.82,0.25,0.23}{##1}}}
\expandafter\def\csname PY@tok@nv\endcsname{\def\PY@tc##1{\textcolor[rgb]{0.10,0.09,0.49}{##1}}}
\expandafter\def\csname PY@tok@no\endcsname{\def\PY@tc##1{\textcolor[rgb]{0.53,0.00,0.00}{##1}}}
\expandafter\def\csname PY@tok@nl\endcsname{\def\PY@tc##1{\textcolor[rgb]{0.63,0.63,0.00}{##1}}}
\expandafter\def\csname PY@tok@ni\endcsname{\let\PY@bf=\textbf\def\PY@tc##1{\textcolor[rgb]{0.60,0.60,0.60}{##1}}}
\expandafter\def\csname PY@tok@na\endcsname{\def\PY@tc##1{\textcolor[rgb]{0.49,0.56,0.16}{##1}}}
\expandafter\def\csname PY@tok@nt\endcsname{\let\PY@bf=\textbf\def\PY@tc##1{\textcolor[rgb]{0.00,0.50,0.00}{##1}}}
\expandafter\def\csname PY@tok@nd\endcsname{\def\PY@tc##1{\textcolor[rgb]{0.67,0.13,1.00}{##1}}}
\expandafter\def\csname PY@tok@s\endcsname{\def\PY@tc##1{\textcolor[rgb]{0.73,0.13,0.13}{##1}}}
\expandafter\def\csname PY@tok@sd\endcsname{\let\PY@it=\textit\def\PY@tc##1{\textcolor[rgb]{0.73,0.13,0.13}{##1}}}
\expandafter\def\csname PY@tok@si\endcsname{\let\PY@bf=\textbf\def\PY@tc##1{\textcolor[rgb]{0.73,0.40,0.53}{##1}}}
\expandafter\def\csname PY@tok@se\endcsname{\let\PY@bf=\textbf\def\PY@tc##1{\textcolor[rgb]{0.73,0.40,0.13}{##1}}}
\expandafter\def\csname PY@tok@sr\endcsname{\def\PY@tc##1{\textcolor[rgb]{0.73,0.40,0.53}{##1}}}
\expandafter\def\csname PY@tok@ss\endcsname{\def\PY@tc##1{\textcolor[rgb]{0.10,0.09,0.49}{##1}}}
\expandafter\def\csname PY@tok@sx\endcsname{\def\PY@tc##1{\textcolor[rgb]{0.00,0.50,0.00}{##1}}}
\expandafter\def\csname PY@tok@m\endcsname{\def\PY@tc##1{\textcolor[rgb]{0.40,0.40,0.40}{##1}}}
\expandafter\def\csname PY@tok@gh\endcsname{\let\PY@bf=\textbf\def\PY@tc##1{\textcolor[rgb]{0.00,0.00,0.50}{##1}}}
\expandafter\def\csname PY@tok@gu\endcsname{\let\PY@bf=\textbf\def\PY@tc##1{\textcolor[rgb]{0.50,0.00,0.50}{##1}}}
\expandafter\def\csname PY@tok@gd\endcsname{\def\PY@tc##1{\textcolor[rgb]{0.63,0.00,0.00}{##1}}}
\expandafter\def\csname PY@tok@gi\endcsname{\def\PY@tc##1{\textcolor[rgb]{0.00,0.63,0.00}{##1}}}
\expandafter\def\csname PY@tok@gr\endcsname{\def\PY@tc##1{\textcolor[rgb]{1.00,0.00,0.00}{##1}}}
\expandafter\def\csname PY@tok@ge\endcsname{\let\PY@it=\textit}
\expandafter\def\csname PY@tok@gs\endcsname{\let\PY@bf=\textbf}
\expandafter\def\csname PY@tok@gp\endcsname{\let\PY@bf=\textbf\def\PY@tc##1{\textcolor[rgb]{0.00,0.00,0.50}{##1}}}
\expandafter\def\csname PY@tok@go\endcsname{\def\PY@tc##1{\textcolor[rgb]{0.53,0.53,0.53}{##1}}}
\expandafter\def\csname PY@tok@gt\endcsname{\def\PY@tc##1{\textcolor[rgb]{0.00,0.27,0.87}{##1}}}
\expandafter\def\csname PY@tok@err\endcsname{\def\PY@bc##1{\setlength{\fboxsep}{0pt}\fcolorbox[rgb]{1.00,0.00,0.00}{1,1,1}{\strut ##1}}}
\expandafter\def\csname PY@tok@kc\endcsname{\let\PY@bf=\textbf\def\PY@tc##1{\textcolor[rgb]{0.00,0.50,0.00}{##1}}}
\expandafter\def\csname PY@tok@kd\endcsname{\let\PY@bf=\textbf\def\PY@tc##1{\textcolor[rgb]{0.00,0.50,0.00}{##1}}}
\expandafter\def\csname PY@tok@kn\endcsname{\let\PY@bf=\textbf\def\PY@tc##1{\textcolor[rgb]{0.00,0.50,0.00}{##1}}}
\expandafter\def\csname PY@tok@kr\endcsname{\let\PY@bf=\textbf\def\PY@tc##1{\textcolor[rgb]{0.00,0.50,0.00}{##1}}}
\expandafter\def\csname PY@tok@bp\endcsname{\def\PY@tc##1{\textcolor[rgb]{0.00,0.50,0.00}{##1}}}
\expandafter\def\csname PY@tok@fm\endcsname{\def\PY@tc##1{\textcolor[rgb]{0.00,0.00,1.00}{##1}}}
\expandafter\def\csname PY@tok@vc\endcsname{\def\PY@tc##1{\textcolor[rgb]{0.10,0.09,0.49}{##1}}}
\expandafter\def\csname PY@tok@vg\endcsname{\def\PY@tc##1{\textcolor[rgb]{0.10,0.09,0.49}{##1}}}
\expandafter\def\csname PY@tok@vi\endcsname{\def\PY@tc##1{\textcolor[rgb]{0.10,0.09,0.49}{##1}}}
\expandafter\def\csname PY@tok@vm\endcsname{\def\PY@tc##1{\textcolor[rgb]{0.10,0.09,0.49}{##1}}}
\expandafter\def\csname PY@tok@sa\endcsname{\def\PY@tc##1{\textcolor[rgb]{0.73,0.13,0.13}{##1}}}
\expandafter\def\csname PY@tok@sb\endcsname{\def\PY@tc##1{\textcolor[rgb]{0.73,0.13,0.13}{##1}}}
\expandafter\def\csname PY@tok@sc\endcsname{\def\PY@tc##1{\textcolor[rgb]{0.73,0.13,0.13}{##1}}}
\expandafter\def\csname PY@tok@dl\endcsname{\def\PY@tc##1{\textcolor[rgb]{0.73,0.13,0.13}{##1}}}
\expandafter\def\csname PY@tok@s2\endcsname{\def\PY@tc##1{\textcolor[rgb]{0.73,0.13,0.13}{##1}}}
\expandafter\def\csname PY@tok@sh\endcsname{\def\PY@tc##1{\textcolor[rgb]{0.73,0.13,0.13}{##1}}}
\expandafter\def\csname PY@tok@s1\endcsname{\def\PY@tc##1{\textcolor[rgb]{0.73,0.13,0.13}{##1}}}
\expandafter\def\csname PY@tok@mb\endcsname{\def\PY@tc##1{\textcolor[rgb]{0.40,0.40,0.40}{##1}}}
\expandafter\def\csname PY@tok@mf\endcsname{\def\PY@tc##1{\textcolor[rgb]{0.40,0.40,0.40}{##1}}}
\expandafter\def\csname PY@tok@mh\endcsname{\def\PY@tc##1{\textcolor[rgb]{0.40,0.40,0.40}{##1}}}
\expandafter\def\csname PY@tok@mi\endcsname{\def\PY@tc##1{\textcolor[rgb]{0.40,0.40,0.40}{##1}}}
\expandafter\def\csname PY@tok@il\endcsname{\def\PY@tc##1{\textcolor[rgb]{0.40,0.40,0.40}{##1}}}
\expandafter\def\csname PY@tok@mo\endcsname{\def\PY@tc##1{\textcolor[rgb]{0.40,0.40,0.40}{##1}}}
\expandafter\def\csname PY@tok@ch\endcsname{\let\PY@it=\textit\def\PY@tc##1{\textcolor[rgb]{0.25,0.50,0.50}{##1}}}
\expandafter\def\csname PY@tok@cm\endcsname{\let\PY@it=\textit\def\PY@tc##1{\textcolor[rgb]{0.25,0.50,0.50}{##1}}}
\expandafter\def\csname PY@tok@cpf\endcsname{\let\PY@it=\textit\def\PY@tc##1{\textcolor[rgb]{0.25,0.50,0.50}{##1}}}
\expandafter\def\csname PY@tok@c1\endcsname{\let\PY@it=\textit\def\PY@tc##1{\textcolor[rgb]{0.25,0.50,0.50}{##1}}}
\expandafter\def\csname PY@tok@cs\endcsname{\let\PY@it=\textit\def\PY@tc##1{\textcolor[rgb]{0.25,0.50,0.50}{##1}}}

\def\PYZbs{\char`\\}
\def\PYZus{\char`\_}
\def\PYZob{\char`\{}
\def\PYZcb{\char`\}}
\def\PYZca{\char`\^}
\def\PYZam{\char`\&}
\def\PYZlt{\char`\<}
\def\PYZgt{\char`\>}
\def\PYZsh{\char`\#}
\def\PYZpc{\char`\%}
\def\PYZdl{\char`\$}
\def\PYZhy{\char`\-}
\def\PYZsq{\char`\'}
\def\PYZdq{\char`\"}
\def\PYZti{\char`\~}
% for compatibility with earlier versions
\def\PYZat{@}
\def\PYZlb{[}
\def\PYZrb{]}
\makeatother


    % Exact colors from NB
    \definecolor{incolor}{rgb}{0.0, 0.0, 0.5}
    \definecolor{outcolor}{rgb}{0.545, 0.0, 0.0}



    
    % Prevent overflowing lines due to hard-to-break entities
    \sloppy 
    % Setup hyperref package
    \hypersetup{
      breaklinks=true,  % so long urls are correctly broken across lines
      colorlinks=true,
      urlcolor=urlcolor,
      linkcolor=linkcolor,
      citecolor=citecolor,
      }
    % Slightly bigger margins than the latex defaults
    
    \geometry{verbose,tmargin=1in,bmargin=1in,lmargin=1in,rmargin=1in}
    
    

    \begin{document}
    
    
    \maketitle
    
    

    
    \section{Indice}\label{indice}

Section \ref{introducao}

Section \ref{desenvolvimento}

Section \ref{resultados}

Section \ref{conclusoes}

Section \ref{bibliografia}

    \section{Introducao}\label{introducao}

A transmissao de sinal e' o processo de enviar e propagar um sinal
analogico ou digital atraves de cabo, fibra optica, \emph{wireless} ou
outro meio de transmissao. Um exemplo de transmissao e' uma chamada
telefonica ou o envio de um e-mail. Tipicamente a transmissao de
informacao e' realizada de forma digital e veio substituir a transmissao
de sinais analogicos, isto porque um sinal analogico nao pode ser limpo
e sofre degradacao em longas distancias, um repetidor analogico pode
amplificar o sinal mas ira tambem amplificar o seu ruido. Os sinais
digitais podem ser reconstruidos com menor ruido e regenerado para o seu
estado original. Estas sao algumas das razoes principais da transmissao
de sinais digitais em grandes distancias por parte das empresas de
telecomunicacoes.

A conversao de analogico para digital e' realizada em tres etapas: A
\textbf{amostragem} que tem como objectivo tornar o sinal discreto no
dominio do tempo e nao envolve perda de informacao respeitando o teorema
de \emph{Nyquist} cuja frequencia do sinal amostrado tem de ser duas
vezes superior a' do sinal original; A \textbf{quantificacao} que torna
as amostras do sinal discretas na amplitude, transformando uma variavel
continua num numero finito de valores; A \textbf{codificacao} que
atribui a cada amplitude de pulso um codigo binario.

O \textbf{controlo de erros} tem como objectivo a deteccao e correccao
de erros, devido a estarmos perante um sinal digital este processo da-se
ao nivel dos codigos binarios do sinal. Existem dois tipos de erros de
bit: os erros devido aos efeitos do canal de natureza aleatoria; e os
erros devido a inteferencias electromagneticas em rajada. Dependendo do
tipo de erros existem estrategias diferenciadas para, no receptor, os
detectar ou mesmo corrigir. Para que seja possivel realizar este
processo e' necessario introduzir bits com informacao relativa ao
controlo de erros.

A \textbf{modelacao digital} e' o processo no qual a informacao
transmitida numa comunicacao e' adicionada ao sinal electromagnetico. O
transmissor adiciona a informacao numa onda especial de tal forma que
podera ser recuparada no receptor atraves de um processo reverso chamado
desmodulacao.

O \textbf{canal} e' meramente o meio utilizado para transmitir o sinal
do transmissor ao receptor. Pode ser um par de condutores, um cabo
coaxial, uma banda de frequencias de radio, um feixe de luz, etc.

\begin{figure}
\centering
\includegraphics{https://preview.ibb.co/eKoeuw/Screenshot_from_2018_01_02_19_00_46.png}
\caption{Envio e recepcao de informacao}
\end{figure}

    \section{Desenvolvimento}\label{desenvolvimento}

A transmissao de sinal e' dividida em varios blocos, a quantizacao,
codificacao, o controlo de erros, a modulacao digital e o envio pelo
canal. No receptor da-se o mesmo processo de forma invertida em cada
bloco. Assim foi construida um biblioteca para isolar o codigo de cada
bloco em diferentes ficheiros.

A \textbf{quantificacao} de um sinal e' o processo que converte um sinal
amostrado, num sinal com valores tambem discretos em amplitude. Para
quantificar um sinal aproxima-se cada valor atraves de um metodo que
atribui os valores de quantificacao \texttt{vj}, para tal usam-se
valores de decisao \texttt{tj} equidistantes que serve de fronteira
entre valores de quantificacao.

\begin{Shaded}
\begin{Highlighting}[]
\KeywordTok{def} \NormalTok{uniform_midrise_quantizer(vmax, delta_q):}
    \NormalTok{vj }\OperatorTok{=} \NormalTok{np.arange(}\OperatorTok{-}\DecValTok{1} \OperatorTok{*} \NormalTok{vmax }\OperatorTok{+} \NormalTok{delta_q }\OperatorTok{/} \DecValTok{2}\NormalTok{, vmax, delta_q)}
    \NormalTok{tj }\OperatorTok{=} \NormalTok{np.arange(}\OperatorTok{-}\DecValTok{1} \OperatorTok{*} \NormalTok{vmax }\OperatorTok{+} \NormalTok{delta_q, vmax, delta_q)}

    \ControlFlowTok{return} \NormalTok{vj, tj}
\end{Highlighting}
\end{Shaded}

\begin{Shaded}
\begin{Highlighting}[]
\KeywordTok{def} \NormalTok{quantize(signal, vmax, vj, tj):}
    \NormalTok{tj }\OperatorTok{=} \NormalTok{np.insert(tj, }\BuiltInTok{len}\NormalTok{(tj), vmax)}

    \CommentTok{# majorate mq array as default value}
    \NormalTok{mq }\OperatorTok{=} \NormalTok{np.ones(}\BuiltInTok{len}\NormalTok{(signal)) }\OperatorTok{*} \NormalTok{np.}\BuiltInTok{max}\NormalTok{(vj)}

    \CommentTok{# majorate the index as default value}
    \NormalTok{idx }\OperatorTok{=} \NormalTok{np.ones(}\BuiltInTok{len}\NormalTok{(signal), dtype}\OperatorTok{=}\StringTok{'uint32'}\NormalTok{) }\OperatorTok{*} \BuiltInTok{len}\NormalTok{(vj)}

    \CommentTok{# loop every point in the signal and check if the point is lower or equal}
    \CommentTok{# than any tj elements (decision values)}
    \ControlFlowTok{for} \NormalTok{i }\KeywordTok{in} \BuiltInTok{range}\NormalTok{(}\BuiltInTok{len}\NormalTok{(signal)):}
        \BuiltInTok{eval} \OperatorTok{=} \NormalTok{signal[i] }\OperatorTok{<=} \NormalTok{tj}

        \CommentTok{# test whether any array element along a given axis evaluates to True}
        \ControlFlowTok{if} \NormalTok{np.}\BuiltInTok{any}\NormalTok{(}\BuiltInTok{eval}\NormalTok{):}
            \NormalTok{xq_value }\OperatorTok{=} \NormalTok{vj[}\BuiltInTok{eval}\NormalTok{][}\DecValTok{0}\NormalTok{]}
            \NormalTok{mq[i] }\OperatorTok{=} \NormalTok{xq_value}

            \CommentTok{# get the index}
            \NormalTok{k }\OperatorTok{=} \NormalTok{np.nonzero(}\BuiltInTok{eval}\NormalTok{)[}\DecValTok{0}\NormalTok{][}\DecValTok{0}\NormalTok{]}
            \NormalTok{idx[i] }\OperatorTok{=} \NormalTok{k}

    \ControlFlowTok{return} \NormalTok{mq, idx}
\end{Highlighting}
\end{Shaded}

\begin{Shaded}
\begin{Highlighting}[]
\KeywordTok{def} \NormalTok{dequantize(vj, indexes):}
    \ControlFlowTok{return} \NormalTok{vj[indexes]}
\end{Highlighting}
\end{Shaded}

    \begin{quote}
\textbf{Exercicio resolvido 3 da sebenta (parte 1, pag. 86)}
\end{quote}

    \begin{Verbatim}[commandchars=\\\{\}]
{\color{incolor}In [{\color{incolor}29}]:} \PY{k+kn}{import} \PY{n+nn}{numpy} \PY{k}{as} \PY{n+nn}{np}
         \PY{k+kn}{from} \PY{n+nn}{lib} \PY{k}{import} \PY{n}{quantization}
         
         \PY{c+c1}{\PYZsh{} quantificar sinais ate 1v}
         \PY{n}{vmax} \PY{o}{=} \PY{l+m+mi}{1}
         
         \PY{c+c1}{\PYZsh{} 8 intervalos de quantificacao}
         \PY{n}{l} \PY{o}{=} \PY{l+m+mi}{8}
         
         \PY{c+c1}{\PYZsh{} intervalo de quantificacao}
         \PY{n}{delta\PYZus{}q} \PY{o}{=} \PY{l+m+mi}{2} \PY{o}{*} \PY{n}{v} \PY{o}{/} \PY{n}{l}
         
         \PY{n}{vj}\PY{p}{,} \PY{n}{tj} \PY{o}{=} \PY{n}{quantization}\PY{o}{.}\PY{n}{uniform\PYZus{}midrise\PYZus{}quantizer}\PY{p}{(}\PY{n}{vmax}\PY{p}{,} \PY{n}{delta\PYZus{}q}\PY{p}{)}
         
         \PY{n+nb}{print}\PY{p}{(}\PY{l+s+s1}{\PYZsq{}}\PY{l+s+s1}{Valores de quantificacao:}\PY{l+s+se}{\PYZbs{}n}\PY{l+s+si}{\PYZob{}\PYZcb{}}\PY{l+s+se}{\PYZbs{}n}\PY{l+s+s1}{\PYZsq{}}\PY{o}{.}\PY{n}{format}\PY{p}{(}\PY{n}{vj}\PY{p}{)}\PY{p}{)}
         \PY{n+nb}{print}\PY{p}{(}\PY{l+s+s1}{\PYZsq{}}\PY{l+s+s1}{Valores de decisao:}\PY{l+s+se}{\PYZbs{}n}\PY{l+s+si}{\PYZob{}\PYZcb{}}\PY{l+s+se}{\PYZbs{}n}\PY{l+s+s1}{\PYZsq{}}\PY{o}{.}\PY{n}{format}\PY{p}{(}\PY{n}{tj}\PY{p}{)}\PY{p}{)}
         
         \PY{n}{n} \PY{o}{=} \PY{n}{np}\PY{o}{.}\PY{n}{arange}\PY{p}{(}\PY{l+m+mi}{0}\PY{p}{,} \PY{l+m+mi}{8}\PY{p}{)}
         \PY{n}{m} \PY{o}{=} \PY{n}{np}\PY{o}{.}\PY{n}{sin}\PY{p}{(}\PY{l+m+mi}{2} \PY{o}{*} \PY{n}{np}\PY{o}{.}\PY{n}{pi} \PY{o}{*} \PY{l+m+mi}{1300} \PY{o}{/} \PY{l+m+mi}{8000} \PY{o}{*} \PY{n}{n}\PY{p}{)}
         \PY{n}{r} \PY{o}{=} \PY{l+m+mi}{3}
         
         \PY{n}{x1}\PY{p}{,} \PY{n}{idx} \PY{o}{=} \PY{n}{quantization}\PY{o}{.}\PY{n}{quantize}\PY{p}{(}\PY{n}{m}\PY{p}{,} \PY{n}{vmax}\PY{p}{,} \PY{n}{vj}\PY{p}{,} \PY{n}{tj}\PY{p}{)}
         
         \PY{n+nb}{print}\PY{p}{(}\PY{l+s+s1}{\PYZsq{}}\PY{l+s+s1}{Quantizacao:}\PY{l+s+se}{\PYZbs{}n}\PY{l+s+si}{\PYZob{}\PYZcb{}}\PY{l+s+se}{\PYZbs{}n}\PY{l+s+s1}{\PYZsq{}}\PY{o}{.}\PY{n}{format}\PY{p}{(}\PY{n}{x1}\PY{p}{)}\PY{p}{)}
         
         \PY{n+nb}{print}\PY{p}{(}\PY{l+s+s1}{\PYZsq{}}\PY{l+s+s1}{Dequantizacao:}\PY{l+s+se}{\PYZbs{}n}\PY{l+s+si}{\PYZob{}\PYZcb{}}\PY{l+s+se}{\PYZbs{}n}\PY{l+s+s1}{\PYZsq{}}\PY{o}{.}\PY{n}{format}\PY{p}{(}\PY{n}{quantization}\PY{o}{.}\PY{n}{dequantize}\PY{p}{(}\PY{n}{vj}\PY{p}{,} \PY{n}{idx}\PY{p}{)}\PY{p}{)}\PY{p}{)}
\end{Verbatim}


    \begin{Verbatim}[commandchars=\\\{\}]
Valores de quantificacao:
[-0.875 -0.625 -0.375 -0.125  0.125  0.375  0.625  0.875]

Valores de decisao:
[-0.75 -0.5  -0.25  0.    0.25  0.5   0.75]

Quantizacao:
[-0.125  0.875  0.875  0.125 -0.875 -0.875 -0.125  0.875]

Dequantizacao:
[-0.125  0.875  0.875  0.125 -0.875 -0.875 -0.125  0.875]


    \end{Verbatim}

    A \textbf{codificacao} e' a representacao binaria da sequencia de
valores de um sinal, onde cada valor e' exprimido atraves de um codigo
binario. A esta codificacao chama-se de modulacao por codigo de pulso
(PCM - \emph{Pulse Code Modulation}). Os indices obtidos no bloco de
quantizacao sao convertidos para o seu respectivo valor binario com uma
representacao a \texttt{r} bits.

O numero de bits representados na codificacao de cada numero e' variavel
em cada problema, entao assumiu-se que o tamanho maximo desta
representacao pode ter um maximo de 32 bits. Para optimizar esta
conversao foi utilizada a funcao
\href{https://docs.scipy.org/doc/numpy-1.13.0/reference/generated/numpy.unpackbits.html}{\texttt{unpackbits()}}
do numpy que converte numeros inteiros de 8 bits. Assim foi criado um
novo tipo de dados que inclui 4 parcelas de numeros a 8 bits, perfazendo
os 32 bits definidos.

\begin{Shaded}
\begin{Highlighting}[]
\KeywordTok{def} \NormalTok{pcm_encode(idx: np.ndarray, r: np.}\BuiltInTok{int}\NormalTok{) }\OperatorTok{->} \NormalTok{np.ndarray:}
    \CommentTok{# new data type to divide an int32 variable into 4 int8 variables}
    \NormalTok{dt }\OperatorTok{=} \NormalTok{np.dtype((np.int32, \{}\StringTok{'f0'}\NormalTok{: (np.uint8, }\DecValTok{3}\NormalTok{), }\StringTok{'f1'}\NormalTok{: (np.uint8, }\DecValTok{2}\NormalTok{), }\StringTok{'f2'}\NormalTok{: (np.uint8, }\DecValTok{1}\NormalTok{), }\StringTok{'f3'}\NormalTok{: (np.uint8, }\DecValTok{0}\NormalTok{)\}))}

    \CommentTok{# convert the vector into new data type}
    \NormalTok{idx_uint8 }\OperatorTok{=} \NormalTok{idx.view(dtype}\OperatorTok{=}\NormalTok{dt)}

    \CommentTok{# pack an numpy array with the 4 uint8 variables}
    \NormalTok{idx_uint8 }\OperatorTok{=} \NormalTok{np.array([idx_uint8[}\StringTok{'f0'}\NormalTok{], idx_uint8[}\StringTok{'f1'}\NormalTok{], idx_uint8[}\StringTok{'f2'}\NormalTok{], idx_uint8[}\StringTok{'f3'}\NormalTok{]])}

    \CommentTok{# transpose so we get each number by row}
    \NormalTok{idx_uint8 }\OperatorTok{=} \NormalTok{np.transpose(idx_uint8)}

    \CommentTok{# convert to binary}
    \NormalTok{idx_bin }\OperatorTok{=} \NormalTok{np.unpackbits(idx_uint8, axis}\OperatorTok{=}\DecValTok{1}\NormalTok{)}

    \CommentTok{# slice into the desired number of bits}
    \NormalTok{idx_bin }\OperatorTok{=} \NormalTok{idx_bin[:, }\BuiltInTok{len}\NormalTok{(idx_bin[}\DecValTok{0}\NormalTok{]) }\OperatorTok{-} \NormalTok{r:}\BuiltInTok{len}\NormalTok{(idx_bin[}\DecValTok{0}\NormalTok{])]}

    \ControlFlowTok{return} \NormalTok{idx_bin}
\end{Highlighting}
\end{Shaded}

\begin{Shaded}
\begin{Highlighting}[]
\KeywordTok{def} \NormalTok{pcm_decode(bits: np.ndarray) }\OperatorTok{->} \NormalTok{np.ndarray:}
    \ControlFlowTok{return} \NormalTok{bits.dot(}\DecValTok{1} \OperatorTok{<<} \NormalTok{np.arange(}\BuiltInTok{len}\NormalTok{(bits[}\DecValTok{0}\NormalTok{]) }\OperatorTok{-} \DecValTok{1}\NormalTok{, }\OperatorTok{-}\DecValTok{1}\NormalTok{, }\OperatorTok{-}\DecValTok{1}\NormalTok{))}
\end{Highlighting}
\end{Shaded}

    \begin{Verbatim}[commandchars=\\\{\}]
{\color{incolor}In [{\color{incolor}32}]:} \PY{k+kn}{from} \PY{n+nn}{lib} \PY{k}{import} \PY{n}{codification}
         
         \PY{n}{x2} \PY{o}{=} \PY{n}{codification}\PY{o}{.}\PY{n}{pcm\PYZus{}encode}\PY{p}{(}\PY{n}{idx}\PY{p}{,} \PY{n}{r}\PY{p}{)}
         
         \PY{n+nb}{print}\PY{p}{(}\PY{l+s+s1}{\PYZsq{}}\PY{l+s+s1}{Indices de quantizacao:}\PY{l+s+se}{\PYZbs{}n}\PY{l+s+si}{\PYZob{}\PYZcb{}}\PY{l+s+se}{\PYZbs{}n}\PY{l+s+s1}{\PYZsq{}}\PY{o}{.}\PY{n}{format}\PY{p}{(}\PY{n}{idx}\PY{p}{)}\PY{p}{)}
         \PY{n+nb}{print}\PY{p}{(}\PY{l+s+s1}{\PYZsq{}}\PY{l+s+s1}{Codificacao:}\PY{l+s+se}{\PYZbs{}n}\PY{l+s+si}{\PYZob{}\PYZcb{}}\PY{l+s+se}{\PYZbs{}n}\PY{l+s+s1}{\PYZsq{}}\PY{o}{.}\PY{n}{format}\PY{p}{(}\PY{n}{x2}\PY{p}{)}\PY{p}{)}
         \PY{n+nb}{print}\PY{p}{(}\PY{l+s+s1}{\PYZsq{}}\PY{l+s+s1}{A sequencia binaria a ser transmitida corresponde a concatenacao }\PY{l+s+s1}{\PYZsq{}} \PYZbs{}
               \PY{l+s+s1}{\PYZsq{}}\PY{l+s+s1}{por ordem temporal dos codigos em binario:}\PY{l+s+se}{\PYZbs{}n}\PY{l+s+si}{\PYZob{}\PYZcb{}}\PY{l+s+se}{\PYZbs{}n}\PY{l+s+s1}{\PYZsq{}}\PY{o}{.}\PY{n}{format}\PY{p}{(}\PY{n}{np}\PY{o}{.}\PY{n}{ndarray}\PY{o}{.}\PY{n}{flatten}\PY{p}{(}\PY{n}{x2}\PY{p}{)}\PY{p}{)}\PY{p}{)}
         \PY{n+nb}{print}\PY{p}{(}\PY{l+s+s1}{\PYZsq{}}\PY{l+s+s1}{Descodificacao de bits:}\PY{l+s+se}{\PYZbs{}n}\PY{l+s+si}{\PYZob{}\PYZcb{}}\PY{l+s+se}{\PYZbs{}n}\PY{l+s+s1}{\PYZsq{}}\PY{o}{.}\PY{n}{format}\PY{p}{(}\PY{n}{codification}\PY{o}{.}\PY{n}{pcm\PYZus{}decode}\PY{p}{(}\PY{n}{x2}\PY{p}{)}\PY{p}{)}\PY{p}{)}
\end{Verbatim}


    \begin{Verbatim}[commandchars=\\\{\}]
Indices de quantizacao:
[3 7 7 4 0 0 3 7]

Codificacao:
[[0 1 1]
 [1 1 1]
 [1 1 1]
 [1 0 0]
 [0 0 0]
 [0 0 0]
 [0 1 1]
 [1 1 1]]

A sequencia binaria a ser transmitida corresponde a concatenacao por ordem temporal dos codigos em binario:
[0 1 1 1 1 1 1 1 1 1 0 0 0 0 0 0 0 0 0 1 1 1 1 1]

Descodificacao de bits:
[3 7 7 4 0 0 3 7]


    \end{Verbatim}

    O \textbf{controlo de erros} atraves do codigo de Hamming consegue
corrigir 1 bit ou detectar 2 bits errados na messagem. Este processo
implica a adicao de bits de paridade a' mensagem para serem utilizados
na correcao.

A correcao e' realizada pelo calculo do sindroma da mensagem que
corresponde ao indice da matrix \texttt{H} e ao indice da mensagem a ser
corrigido.

\begin{Shaded}
\begin{Highlighting}[]
\KeywordTok{def} \NormalTok{hamming(x: np.ndarray, P: np.ndarray, n: np.}\BuiltInTok{int}\NormalTok{, r: np.}\BuiltInTok{int}\NormalTok{) }\OperatorTok{->} \NormalTok{np.ndarray:}
    \NormalTok{G }\OperatorTok{=} \NormalTok{np.hstack((np.identity(r, dtype}\OperatorTok{=}\StringTok{'uint8'}\NormalTok{), P))}

    \ControlFlowTok{return} \NormalTok{np.dot(x, G) }\OperatorTok \DecValTok{2}

    \CommentTok{# if S is == 0 then there's no error in that sub-message}
    \CommentTok{# else find the row position where sub-message is equal to S}
    \CommentTok{# then flip the bit in the sub-message at row position}
    \ControlFlowTok{for} \NormalTok{row }\KeywordTok{in} \BuiltInTok{range}\NormalTok{(}\BuiltInTok{len}\NormalTok{(S)):}
        \ControlFlowTok{if} \NormalTok{np.}\BuiltInTok{all}\NormalTok{(S[row] }\OperatorTok{==} \DecValTok{0}\NormalTok{):}
            \ControlFlowTok{continue}

        \NormalTok{col }\OperatorTok{=} \NormalTok{np.argwhere(np.}\BuiltInTok{all}\NormalTok{(H }\OperatorTok{==} \NormalTok{S[row], axis}\OperatorTok{=}\DecValTok{1}\NormalTok{))[}\DecValTok{0}\NormalTok{][}\DecValTok{0}\NormalTok{]}
        \NormalTok{y[row, col] }\OperatorTok{=} \NormalTok{np.logical_not(y[row, col])}

    \ControlFlowTok{return} \NormalTok{y[:, }\DecValTok{0}\NormalTok{:}\OperatorTok{-}\DecValTok{1} \OperatorTok{*} \BuiltInTok{len}\NormalTok{(P[}\DecValTok{0}\NormalTok{])]}
\end{Highlighting}
\end{Shaded}

    \begin{quote}
\textbf{Exercicio resolvido em aula}
\end{quote}

    \begin{Verbatim}[commandchars=\\\{\}]
{\color{incolor}In [{\color{incolor}40}]:} \PY{k+kn}{from} \PY{n+nn}{lib} \PY{k}{import} \PY{n}{error\PYZus{}control}
         
         \PY{c+c1}{\PYZsh{} mensagem}
         \PY{n}{m} \PY{o}{=} \PY{p}{[}\PY{p}{[}\PY{l+m+mi}{0}\PY{p}{,} \PY{l+m+mi}{1}\PY{p}{,} \PY{l+m+mi}{1}\PY{p}{,} \PY{l+m+mi}{1}\PY{p}{]}\PY{p}{,} \PY{p}{[}\PY{l+m+mi}{0}\PY{p}{,} \PY{l+m+mi}{0}\PY{p}{,} \PY{l+m+mi}{1}\PY{p}{,} \PY{l+m+mi}{0}\PY{p}{]}\PY{p}{]}
         
         \PY{c+c1}{\PYZsh{} matriz paridade}
         \PY{n}{p} \PY{o}{=} \PY{p}{[}\PY{p}{[}\PY{l+m+mi}{0}\PY{p}{,} \PY{l+m+mi}{1}\PY{p}{,} \PY{l+m+mi}{1}\PY{p}{]}\PY{p}{,} \PY{p}{[}\PY{l+m+mi}{1}\PY{p}{,} \PY{l+m+mi}{1}\PY{p}{,} \PY{l+m+mi}{0}\PY{p}{]}\PY{p}{,} \PY{p}{[}\PY{l+m+mi}{1}\PY{p}{,} \PY{l+m+mi}{0}\PY{p}{,} \PY{l+m+mi}{1}\PY{p}{]}\PY{p}{,} \PY{p}{[}\PY{l+m+mi}{1}\PY{p}{,} \PY{l+m+mi}{1}\PY{p}{,} \PY{l+m+mi}{1}\PY{p}{]}\PY{p}{]}
         
         \PY{n}{n} \PY{o}{=} \PY{l+m+mi}{7}
         \PY{n}{k} \PY{o}{=} \PY{l+m+mi}{4}
         
         \PY{n}{x4} \PY{o}{=} \PY{n}{error\PYZus{}control}\PY{o}{.}\PY{n}{hamming}\PY{p}{(}\PY{n}{m}\PY{p}{,} \PY{n}{p}\PY{p}{,} \PY{n}{n}\PY{p}{,} \PY{n}{k}\PY{p}{)}
         
         \PY{n+nb}{print}\PY{p}{(}\PY{l+s+s1}{\PYZsq{}}\PY{l+s+s1}{Mensagem com bits de paridade:}\PY{l+s+se}{\PYZbs{}n}\PY{l+s+si}{\PYZob{}\PYZcb{}}\PY{l+s+se}{\PYZbs{}n}\PY{l+s+s1}{\PYZsq{}}\PY{o}{.}\PY{n}{format}\PY{p}{(}\PY{n}{np}\PY{o}{.}\PY{n}{ndarray}\PY{o}{.}\PY{n}{flatten}\PY{p}{(}\PY{n}{x4}\PY{p}{)}\PY{p}{)}\PY{p}{)}
         \PY{n+nb}{print}\PY{p}{(}\PY{l+s+s1}{\PYZsq{}}\PY{l+s+s1}{Mensagem sem bits de paridade:}\PY{l+s+se}{\PYZbs{}n}\PY{l+s+si}{\PYZob{}\PYZcb{}}\PY{l+s+se}{\PYZbs{}n}\PY{l+s+s1}{\PYZsq{}}\PY{o}{.}\PY{n}{format}\PY{p}{(}\PY{n}{np}\PY{o}{.}\PY{n}{ndarray}\PY{o}{.}\PY{n}{flatten}\PY{p}{(}\PY{n}{error\PYZus{}control}\PY{o}{.}\PY{n}{correction}\PY{p}{(}\PY{n}{x4}\PY{p}{,} \PY{n}{p}\PY{p}{)}\PY{p}{)}\PY{p}{)}\PY{p}{)}
\end{Verbatim}


    \begin{Verbatim}[commandchars=\\\{\}]
Mensagem com bits de paridade:
[0 1 1 1 1 0 0 0 0 1 0 1 0 1]

Mensagem sem bits de paridade:
[0 1 1 1 0 0 1 0]


    \end{Verbatim}

    \section{Resultados}\label{resultados}

\begin{itemize}
\item
  Resultados produzidos com a simulacao
\item
  Correr o lab04 passo a passo, analisar e comentar os resultados
\end{itemize}

    \begin{Verbatim}[commandchars=\\\{\}]
{\color{incolor}In [{\color{incolor}14}]:} \PY{k+kn}{from} \PY{n+nn}{lab04} \PY{k}{import} \PY{n}{lab04}
\end{Verbatim}


    \section{Conclusoes}\label{conclusoes}

\begin{itemize}
\tightlist
\item
  O que correu bem
\item
  Optimizacoes possiveis
\item
  Alteracoes futuras
\item
  Casos praticos
\item
  Simulacao virtual vs simulacao fisica
\end{itemize}

    \section{Bibliografia}\label{bibliografia}

\begin{itemize}
\tightlist
\item
  Sebenta CPS?
\end{itemize}


    % Add a bibliography block to the postdoc
    
    
    
    \end{document}
